\documentclass{article}
	\usepackage{cite}
	\usepackage{graphicx}
	\usepackage{hyperref}
	\usepackage[utf8]{inputenc}
	\usepackage{textcomp}
	\usepackage{wrapfig}

\title{Basidiomycete immutability and integrity: --80~\textdegree C cryovials and ITS sequences}
\author{Peter J. Collins
	\footnote{Please address correspondence to
	\href{mailto:desk@pjc.is}{desk@pjc.is}.}}
\date{\today}

\begin{document}
	\frenchspacing
	\maketitle


\begin{abstract}

Homolka's perlite protocol~\cite{homolka2001, homolka2006} is a clean and simple method to store basidiomycetes.
This paper describes my implementation of the soaked perlite method and experiences with fungal barcoding.

\textbf{Keywords:}
	bioinformatics,
	cryopreservation,
	gene library,
	phylogenetic DNA barcodes

\end{abstract}


\section{Introduction}

Homolka surveyed basidiomycete storage techniques in 2013 with increasing acclaim: agar slants with and without paraffin, water and saline, drying and freeze-drying, and liquid nitrogen.~\cite{homolka2014}
Each element has its place in a managed culture library optimized for data integrity.

Agar slants provide the researcher-cultivator with working inoculum.
Saline keeps at room temperature and thereby increases the library's resilience.
In computing terms, they're RAM and hard disks, respectively.

Cryovials with ITS sequences represent tape backups with file hashes.
Filamentous fungi tend not to survive frozen liquids or freeze drying as well as bacteria, mold, and spores.

Higher fungi typically don't sporulate again until they form fruitbodies.
Producing a sexual organ requires an exponential increase in vegetative growth.
Perlite protocol makes it easy to isolate and store dikaryotic fungi soon after spore germination.

This resolves a chicken-and-egg dilemma in mushroom cultivation.


\section{Materials and methods}

The ITS-sequenced cryovials supplement other copies of my culture library in agar slants and saline vials.
First I fermented a 10\% glycerol broth with antibiotics and perlite, then I stored the drained crystals.

I returned to the source plates to extract, amplify, and verify the ITS gene data.
Then I sent the results to an external lab for Sanger sequencing.


\subsection{Preparing the cryovials}

\begin{wrapfigure}{r}{0.5\textwidth}
	\includegraphics[width=0.5\textwidth]{cryovials}
	\caption{First preparation of Homolka's perlite protocol}
\end{wrapfigure}

I made an antibiotic cryoprotectant medium with Lennox LB (5 g/L NaCl), 25 $\mu$g/L chloramphenicol, and 10\% glycerol.
Then I aerobically fermented each isolate for 10 days with 0.5g perlite in 5 mL broth.
The shaking incubator ran at 28~\textdegree C and 150 RPM.

I drained the fermentation vials of liquid and transferred the inoculated perlite to 2 mL cryovials.
Before discarding the LB and storing the perlite, I viewed representative samples under a microscope.
\emph{G. sessile} and \emph{P. nameko} showed hyphae and conidia at 40$\times$ magnification.

I stored the cryovials in a cardboard freezer box and chilled them, stepping down to control the cooling rate: --20~\textdegree C until frozen, then --80~\textdegree C.


\subsection{Generating ITS data}

%Rockefeller's DNA sequencing protocol is purpose-made to facilite ITS sequencing in a DIYbio setting.~\cite{rockefeller}

%I collected mycelium tissue samples from colonized MEA plates by gently scraping the strongest growth areas with sterile toothpicks.
%Then I manually ground the tissue in 1.5 mL vials with 200 $\mu$L  0.5M NaOH after Wang et al.~\cite{wang1993}
%I pipetted 5 $\mu$L DNA extract in fresh vials with 495 $\mu$L 100 mM Tris--HCl buffer (pH 8.0) and used 1 $\mu$L solution per rection.

%The PCR volume was 25 $\mu$L: 1 $\mu$L DNA extract, 0.5 $\mu$L each ITS1F and ITS4 forward and reverse primers, 5 $\mu$L [ADD BRAND] 5$\times$ PCR master mix, and 18 $\mu$L nuclease-free water.
%The reaction finished in 1:41:02.

\begin{center}
	\begin{tabular}{l|ccc}

% Headers
\textbf{Phase} & \textbf{Time} & \textbf{Temperature} & \textbf{Cycles} \\
\hline

% Data
Denaturation I & 120s & 95~\textdegree C & 1 \\
Denaturation II & 30s & 95~\textdegree C & 30 \\
Annealing & 30s & 54~\textdegree C & 30 \\
Extension & 55s & 72~\textdegree C & 30 \\
Cooling & $\infty$ & 4~\textdegree C & nil \\

	\end{tabular}
\end{center}

The positive control was a pGreen plasmid~\cite{hellens2000} and the negative control was nuclease-free water.
I ran gel electrophoresis on a miniPCR blueGel system with a 1 kb ladder.
I also ran a Qubit fluorometer with [ADD PARAMETERS].

\textsc{todo.} Run the PCR again, decide on Qubit or gel, and send samples for Sanger sequencing.


\section{Results and discussion}

The --80~\textdegree C freezer underscores a larger storage strategy.
Ideally the cryovials should never be opened except as a last resort.

Working quantities of mycelium are more readily available in agar slants or saline vials.
Cold storage with ITS sequences are a trusted baseline to return to in case I compromise any downstream samples.


\subsection{1st generation results}

\begin{center}
	\begin{tabular}{l|ccc}

% Headers
\textbf{Species} & \textbf{P Value} & \textbf{Isolation date} & \textbf{Viability} \\
\hline

% Data
\emph{A. aegerita} & P-5 & 2018-Mar-01 & nil \\
\hline

\emph{G. lucidum} & P-1 & 2018-Mar-01 & nil \\
\emph{G. sessile} & P-1 & 2018-Mar-01 & nil \\
\hline

\emph{G. frondosa} \textsc{nh} & P-1 & 2018-Mar-01 & nil \\
\hline

\emph{H. abietis} & P-4 & 2018-Mar-01 & nil \\
\emph{H. americanum} & P-2 & 2018-Mar-01 & nil \\
\hline

\emph{I. obliquus} & P-1 & 2018-Mar-01 & nil \\
\hline

\emph{I. resinosum} & P-1 & 2018-Mar-01 & nil \\
\hline

\emph{L. edodes} & P-4 & 2018-Mar-01 & nil \\
\hline

\emph{P. roqueforti} & P-2 & 2018-Mar-01 & nil \\
\hline

\emph{P. nameko} & P-1 & 2018-Mar-01 & nil \\
\hline

\emph{P. tuber-regium} & P-2 & 2018-Mar-01 & nil \\

	\end{tabular}
\end{center}


\subsection{Fine-tuning the medium}

Successful cryopreservation is largely a function of freezing and revival times: slow to freeze, quick to revive.

Malt extract is the typical basal sugar used in mushroom cultivation because its low pH and high peptone content selectively favors fungi over bacteria.
A simplified version of Czapek medium may be ideal for perlite protocol: 30 g/L malt extract, 10\% glycerol, 6 g/L sea salt, and an optional antibiotic.

Assuming negligible or no metabolism at --80~\textdegree C, using 50\% more sugar than standard MEA (20 g/L) may speed up revival without risking growth or contamination in storage.

A similarly high sugar ratio wouldn't likely be suitable for agar slants that require gas exchange for a slower but active metabolism.
Serial transfers with agar slants excel on less nutritious media such as cornmeal to prevent metabolism under refrigeration.~\cite{GGMM}


\section{Future work}

Anthropogenic mass extinction fundamentally changes the environment and accelerates the rate of random mutations.
Therefore it's necessary to guarantee the persistence and integrity of as many beneficial genomes as possible.


\subsection{Reconstructing helpful biomolecules}

\textsc{todo.}


\subsection{Developing proprietary species variants}

Developing a mushroom variety typically involves rigorous phenotype selection over 7--10 generations of spore-to-shroom grows.
This rewards the cultivator with a reliable source of novel genetic lineages that mostly express the same desired traits.
The presence or absence of metabolites is another useful selection bias that benefits from comparing a diverse population.

A good genetic balance is one print to many germination plates.
Take spores from the model specimen of your crop to enforce a higher minimum standard and advance the lineage.
Make as many germination plates as possible to ensure that your starting material represents the fittest progeny in a diverse population.

Collect spores from the desired sample to iterate your selections and advance the lineage of the variety.
Agar slants of reference isolates provide a source of inoculum for aerobically fermented malt or dextrose broth.
This generates enough mycelium to inoculate, colonize, and birth sterile sawdust and wheat bran cakes in a humid terrarium.
The resulting mushrooms provide spores for the next generation and food for the cultivator.


% References

\bibliographystyle{plain}
\bibliography{biblio}

\end{document}
